%% -----------------------------------------------------------------------------
%% MIT License
%%
%% Copyright 2025 ThirtySomething
%%
%% Permission is hereby granted, free of charge, to any person obtaining a
%% copy of this software and associated documentation files (the "Software"),
%% to deal in the Software without restriction, including without limitation
%% the rights to use, copy, modify, merge, publish, distribute, sublicense,
%% and/or sell copies of the Software, and to permit persons to whom the
%% Software is furnished to do so, subject to the following conditions:
%%
%% The above copyright notice and this permission notice shall be included
%% in all copies or substantial portions of the Software.
%%
%% THE SOFTWARE IS PROVIDED "AS IS", WITHOUT WARRANTY OF ANY KIND, EXPRESS
%% OR IMPLIED, INCLUDING BUT NOT LIMITED TO THE WARRANTIES OF MERCHANTABILITY,
%% FITNESS FOR A PARTICULAR PURPOSE AND NONINFRINGEMENT. IN NO EVENT SHALL
%% THE AUTHORS OR COPYRIGHT HOLDERS BE LIABLE FOR ANY CLAIM, DAMAGES OR OTHER
%% LIABILITY, WHETHER IN AN ACTION OF CONTRACT, TORT OR OTHERWISE, ARISING
%% FROM, OUT OF OR IN CONNECTION WITH THE SOFTWARE OR THE USE OR OTHER
%% DEALINGS IN THE SOFTWARE.
%% -----------------------------------------------------------------------------

%% -----------------------------------------------------------------------------
%% Scenario I for the usage of Syncthing
%% -----------------------------------------------------------------------------

\section{Einsatzszenario}

Nehmen wir einmal an, dass in einer kleinen Firma eine Bürofrachkraft die
Korrespondenz erledigt. Dabei werden täglich mehrere, unter Umständen dutzende
Schriftstücke erstellt. Bedingt durch die Größe der Firma gibt es keinen
Server, auf dem die Daten zentral abgelegt werden. Stattdessen findet eine
sogenannte lokale Speicherung auf dem Bürocomputer der Fachkraft statt.

\subsection{Technische Redundanz Büro}

\subsubsection{Voraussetzungen}

Bevor mit der Einrichtung von \gls{syncthing} begonnen werden kann, sollte
vom Bürocomputer eine Datensicherung der Dokumente auf die externe USB
Festplatte erfolgen.

\subsubsection{Einrichtung NAS Büro}

Zunächst wird die \gls{ds124} im Büro vorbereitet und \gls{syncthing}
installiert. Eine Anleitung dazu ist
\href{https://github.com/ThirtySomething/NAS-Information}{hier}
zu finden. Bitte darauf achten, dass der Container

\begin{itemize}
    \item Automatisch gestartet wird.
    \item Mindestens beim Speicher limitiert wird; bei der \gls{ds124} werden
          512MB empfohlen.
\end{itemize}

Sollten diese Punkte nicht umgesetzt werden, ist das ganze Konzept nichtig.
Unter Umständen würde vom Betriebssystem des \gls{nas} der \gls{docker}
Container von \gls{syncthing} beendet werden, weil er zu viel Speicher benötigt.
Ohne die Autostart-Option würde er nicht mehr gestartet werden -- und damit
würde die Idee der technischen Redundanz nicht mehr funktionieren.

\subsubsection{Einrichtung PC Büro}

Auf dem Bürocomputer der Fachkraft wird \gls{syncthing} installiert. Sofern
eine Anzeige gewünscht wird, kann optional \gls{syncthingtray}
installiert werden. Das erleichtert die visuelle Kontrolle über den Prozess.
\bigskip

Im nächsten Schritt müssen die \gls{syncthing} Instanzen des Bürocomputers und
der \gls{ds124} miteinander bekannt gemacht werden. Das ist auf der
\href{https://docs.syncthing.net/intro/getting-started.html}{Seite} des
Projektes beschrieben. Die Oberfläche dazu ruft man mit
\href{http://localhost:8384/}{http://localhost:8384/} im Browser auf. Unter
Umständen funktioniert diese URL nicht direkt. Das kann daran liegen, dass bei
der Installation ein Sicherheitszertifikat installiert wurde. Dann ist die
Konfigurationsseite über \href{https://localhost:8384/}{https://localhost:8384/}
zu erreichen. \tsTextBold{Hinweis:} Unter Umständen muss im Browser erst eine
Sicherheitswarnung akzeptiert werden, da das Sicherheitszertifikat während
der Installation erzeugt wurde und dem Broswer nicht bekannt ist.
\bigskip

Danach werden einer oder mehere Ordner auf dem Bürocomputer ausgewählt und in
Syncthing aufgenommen. Auch hier ist die Oberfläche im Browser hilfreich. Es
können alle Ordner auf dem Bürocomputer ausgewählt werden. Eventuell macht es
Sinn, hier eine neue Ordnerstruktur anzulegen/definieren, und dann nach und
nach die originalen Dokumente dorthin zu verschieben.
\bigskip

Nachdem auf dem Bürocomputer die zu synchronisierenden Ordner definiert wurden,
müssen diese nur noch mit der \gls{syncthing} Instanz auf dem \gls{nas}
geteilt werden.

\tsImage{images/es_001a.png}{Technische Redundanz Büro}{width=0.9\textwidth}

Über die Oberfläche des \gls{nas} im Büro akzeptiert man den eingehenden
Ordner des Büro PCs. Damit ist die Einrichtung der technische Redundanz
im Büro abgeschlossen.
\bigskip

\tsTextBold{Hinweis:} Je nach Anzahl Dokumente sowie deren Umfang kann der
initiale Abgleich etwas längere Zeit in Anspruch nehmen.
\bigskip

\tsTextBold{Anmerkung:} Der aktive Datenbestand ist nun auf dem Bürocomputer
und dem \gls{nas} redundant vorhanden.

\subsection{Stufe 2: Versionierung}
\label{sec:Stufe 2: Versionierung}

Die Versionierung ist eine Eigenschaft der Ordner, die mit \gls{syncthing}
geteilte wurde. Diese Eigenschaft ist auf dem Gerät zu aktivieren und
einzustellen, auf dem die Versionierung gewünscht ist.
\bigskip

Die Versionierung soll auf dem \gls{nas} aktiviert werden. Dazu ruft man auf dem
\gls{nas} die Oberfläche von \gls{syncthing} über die IP Adresse des \gls{nas}
mit dem Port 8384 auf: \href{http://IP-DES-NAS:8384/}{http://IP-DES-NAS:8384/}
bzw. \href{https://IP-DES-NAS:8384/}{https://IP-DES-NAS:8384/}.
\bigskip

Hier kann man den Ordner in der Liste links auswählen. Durck einen Mausklick
werden die Details dazu eingeblendet, einschließlich verschiedener
Aktionsbuttons im unteren Bereich. Der notwendige Dialog, um die Versionierung
zu aktivieren und zu konfigurieren, wird über den Bearbeiten bzw. Edit Button
aufgerufen.
\bigskip

Im Dialog wird Versionierung bzw. Versioning ausgewählt.

\tsImage{images/versioning.png}{Versionierungseinstellungen}{width=0.9\textwidth}

In dem abgebildeten Beispiel ist die Versionierung aktiv, es werden maximal
drei Versionen einer Datei vorgehalten. Dies würde dem \gls{generationenprinzip}
entsprechen.
\bigskip

In der \href{https://docs.syncthing.net/v1.29.7/users/versioning}{Hilfe} von
\gls{syncthing} wird näher auf die Versionierung eingegangen. Sollten andere
Einstellungen gewünscht sein, bitte hier genau nachlesen, was die einzelnen
Versionierungsarten bedeuten.

\tsImage{images/es_001b.png}{Versionierung}{width=0.9\textwidth}

Damit ist die Einrichtung des Versionierung abgeschlossen.
\bigskip

\tsTextBold{Anmerkung:} Der aktive Datenbestand auf dem \gls{nas} wird durch
das Programm \gls{syncthing} nicht nur synchronisiert, sondern durch die
Versionierung auch noch in einen passiven Datenbestand überführt.

\subsection{Stufe 3: Logische Redundanz}

Für die logische Redundanz wird ein zweites \gls{nas} an einem anderen
physischen Ort benötigt. Der Einfachkeit halber nehmen wir an, dass dies bei
dem Firmeninhaber \tsTextQuoteG{Zuhause} sein wird.
\bigskip

Das zweite \gls{nas} \tsTextQuoteG{Zuhause} wird analog dem ersten \gls{nas} im
Büro eingerichtet. Dies betrifft sowohl die statische IP Adresse als auch den
\gls{docker} Container von \gls{syncthing}.
\bigskip

Für die logische Redundanz müssen die \gls{syncthing} Instanzen der \gls{nas}
Systeme miteinander bekannt gemacht werden. Das geschieht analog wie bei
den \gls{syncthing} Instanzen des Büro Computers und des \gls{nas} im Büro.
\bigskip

Im letzten Schritt müssen nun die Daten von dem \gls{nas} im Büro auf das
\gls{nas} \tsTextQuoteG{Zuhause} übertragen werden. Dazu geht man ähnlich vor
wie bei der Einstellung der Versionierung. Auf dem \gls{nas} im Büro wählt man
den Reiter Teilen/Share aus. Hier aktiviert man lediglich das Auswahlkästchen
für das \gls{nas} \tsTextQuoteG{Zuhause}. Über die Oberfläche des \gls{nas}
\tsTextQuoteG{Zuhause} akzeptiert man den eingehenden Ordner von dem \gls{nas}
Büro.

\tsImage{images/es_001c.png}{Logische Redundanz}{width=0.9\textwidth}

Soll die Versionierung auch für das \gls{nas} \tsTextQuoteG{Zuhause} gelten,
ist diese auf dem Gerät analog zu \nameref{sec:Stufe 2: Versionierung}
einzurichten. Damit ist die Einrichtung der logsichen Redundanz abgeschlossen.
\bigskip

\tsTextBold{Hinweis:} Je nach Anzahl Dokumente sowie deren Umfang kann der
initiale Abgleich etwas längere Zeit in Anspruch nehmen.

\subsection{Stufe 4: Technische Redundanz Zuhause (optional)}

Möchte der Geschäftsinhaber die Möglichkeit haben, auch zuhause an Dokumenten
zu Arbeiten und diese komfortabel auf dem Büro PC verfügbar haben, kann optional
noch eine Technische Redundanz zwischen dem \gls{nas} \tsTextQuoteG{Zuhause}
und einem PC \tsTextQuoteG{Zuhause} eingerichtet werden.
\bigskip

Dazu installiert man die Programme \gls{syncthing} und \gls{syncthingtray}
auf dem PC \tsTextQuoteG{Zuhause}. Dann macht man die \gls{syncthing} Instanzen
des \gls{nas} \tsTextQuoteG{Zuhause} und PC \tsTextQuoteG{Zuhause} miteinander
bekannt. Im letzten Schritt teilt man den Ordner von dem \gls{nas}
\tsTextQuoteG{Zuhause} mit dem PC \tsTextQuoteG{Zuhause}. Bei der Annahme des
geteilten Ordners ist darauf zu achten, dass der Zielordner, also der Platz,
an dem die Daten abgelegt werden, entsprechend den Wünschen/Vorlieben ausgewählt
wird. \tsTextBold{Hinweis:} Je nach Auswahl des Zielordners ist unter Umständen
ein vorheriges Backup auf einer USB Festplatte angebracht.

\tsImage{images/es_001d.png}{Technische Redundanz Zuhause}{width=0.9\textwidth}

Damit ist die technische Redundanz \tsTextQuoteG{Zuhause} eingerichtet.
\bigskip

\tsTextBold{Hinweis:} Je nach Anzahl Dokumente sowie deren Umfang kann der
initiale Abgleich etwas längere Zeit in Anspruch nehmen.
