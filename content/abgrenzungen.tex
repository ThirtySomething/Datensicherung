%% -----------------------------------------------------------------------------
%% MIT License
%%
%% Copyright 2025 ThirtySomething
%%
%% Permission is hereby granted, free of charge, to any person obtaining a
%% copy of this software and associated documentation files (the "Software"),
%% to deal in the Software without restriction, including without limitation
%% the rights to use, copy, modify, merge, publish, distribute, sublicense,
%% and/or sell copies of the Software, and to permit persons to whom the
%% Software is furnished to do so, subject to the following conditions:
%%
%% The above copyright notice and this permission notice shall be included
%% in all copies or substantial portions of the Software.
%%
%% THE SOFTWARE IS PROVIDED "AS IS", WITHOUT WARRANTY OF ANY KIND, EXPRESS
%% OR IMPLIED, INCLUDING BUT NOT LIMITED TO THE WARRANTIES OF MERCHANTABILITY,
%% FITNESS FOR A PARTICULAR PURPOSE AND NONINFRINGEMENT. IN NO EVENT SHALL
%% THE AUTHORS OR COPYRIGHT HOLDERS BE LIABLE FOR ANY CLAIM, DAMAGES OR OTHER
%% LIABILITY, WHETHER IN AN ACTION OF CONTRACT, TORT OR OTHERWISE, ARISING
%% FROM, OUT OF OR IN CONNECTION WITH THE SOFTWARE OR THE USE OR OTHER
%% DEALINGS IN THE SOFTWARE.
%% -----------------------------------------------------------------------------

%% -----------------------------------------------------------------------------
%% Delimitation
%% -----------------------------------------------------------------------------

\section{Abgrenzungen}

Ein \gls{server} ist ein sehr leistungsfähiger Computer, der im Regelfall rund
um die Uhr läuft. Er stellt verschiedene Dienste im Netzwerk für die Nutzer
bereit. Dazu zählen beispielsweise Festplattenspeicher, Druckerdienste oder
auch Mailserver.
\bigskip

Ein \gls{nas} ist ein energiesparendes Computersystem, welches ebenfalls rund
um die Uhr läuft. Die Hauptaufgabe solch eines Systems ist es, Festplattenspeicher
im Netzwerk für andere Geräte bereitszustellen.
\bigskip

Durch das Fortschreiten der Technologie, sowohl der Hardware als auch der
Software, verschwimmen die Grenzen. Somit können heutige \gls{nas} durchaus
auch als \gls{server} genutzt werden. Bestimmte Dinge sind damit zwar nicht
einfach oder gar nicht umsetzbar, wie beispielsweise die Funktion eines
\gls{domain controller}s. Kleinere Unternehmen können aber auch ohne diese
Funktionalität auskommen.
