%% -----------------------------------------------------------------------------
%% MIT License
%%
%% Copyright 2025 ThirtySomething
%%
%% Permission is hereby granted, free of charge, to any person obtaining a
%% copy of this software and associated documentation files (the "Software"),
%% to deal in the Software without restriction, including without limitation
%% the rights to use, copy, modify, merge, publish, distribute, sublicense,
%% and/or sell copies of the Software, and to permit persons to whom the
%% Software is furnished to do so, subject to the following conditions:
%%
%% The above copyright notice and this permission notice shall be included
%% in all copies or substantial portions of the Software.
%%
%% THE SOFTWARE IS PROVIDED "AS IS", WITHOUT WARRANTY OF ANY KIND, EXPRESS
%% OR IMPLIED, INCLUDING BUT NOT LIMITED TO THE WARRANTIES OF MERCHANTABILITY,
%% FITNESS FOR A PARTICULAR PURPOSE AND NONINFRINGEMENT. IN NO EVENT SHALL
%% THE AUTHORS OR COPYRIGHT HOLDERS BE LIABLE FOR ANY CLAIM, DAMAGES OR OTHER
%% LIABILITY, WHETHER IN AN ACTION OF CONTRACT, TORT OR OTHERWISE, ARISING
%% FROM, OUT OF OR IN CONNECTION WITH THE SOFTWARE OR THE USE OR OTHER
%% DEALINGS IN THE SOFTWARE.
%% -----------------------------------------------------------------------------

%% -----------------------------------------------------------------------------
%% Backup strategies
%% -----------------------------------------------------------------------------

\section{Strategien}

Ein Backup soll einem Datenverlust vorbeugen. Hierfür sind verschiedene
Strategien und Vorkehrungen zu treffen, um im Fall der Fälle wieder einen
Zugriff auf die digitalen Daten zu haben. Hierfür gibt es verschiedene
Möglichkeiten, die entsprechend bewertet werden müssen und in die Backupstrategie
einfliessen sollten.

\subsection{Technische Redundanz}

Solange die Daten nur auf einer einzelnen \gls{festplatte} gespeichert werden,
besteht das Risiko eines 100\%igen Datenverlustes, wenn diese \gls{festplatte}
durch einen technischen Defekt ihre Funktionalität verliert und auf die Daten
nicht mehr zugegriffen werden können.
\bigskip

In der IT wurde dieses Problem bereits Mitte/Ende der 1980er Jahre erkannt. Um
einem 100\%igen Datenverlust vorzubeugen, hatte man die Idee, mehrere
\gls{festplatte}n in einem Verbund zu betreiben, einem sogenannten \gls{raid}.
Hier sorgt also eine spezielle Hardware bzw. eine spezielle Software im
Betriebssystem dafür, dass die Daten redundant vorliegen. Sollte also eine der
\gls{festplatte}n im Verbund ausfallen, entsteht kein Datenverlust.

\subsection{Logische Redundanz}

Trotz einer technischen Redundanz kann ein 100\%iger Datenverlust erfolgen,
wenn das Gerät mit dem \gls{raid} physisch zerstört wird. Das kann zum Beispiel
die Folge von einem Brand oder einem Erdbeben sein. Daher ist es die logische
Konsequenz, die Daten an verschiedenen Orten gleichzeitig vorzuhalten. Früher
hatte man die Bänder/\gls{festplatte}n der Datensicherung nach dem Backup an
einem weiteren Ort gelagert. Heute kann man die Daten über das Internet bequem
auf einem/mehreren weiteren System(en) speichern.

\subsection{Generationenprinzip}

Bei den bisher angerissenen Strategien ging es nur um die aktuellen Daten, die
Just-In-Time redundant, unter Umständen auch über das Internet, vorlagen. Was
aber, wenn ein Dokument versehentlich gelöscht wurde? Und was, wenn diese
Löschung erst nach mehreren Tagen bemerkt wird? Mit den bisherigen Strategien
wäre trotz aller Vorsichtsmaßnahmen ein Datenverlust entstanden. Daraus ergeben
sich zwei wichtige Lehren für die Zukunft:

\begin{itemize}
    \item [1.] Ein \gls{raid} ist keine Datensicherung!
    \item [2.] Eine logische Redundanz schliesst einen Datenverlust nicht aus!
\end{itemize}

Eine Lösung hierfür ist das \gls{generationenprinzip}. Früher wurde eine
Generation durch den Wechsel des Datensicherungsmediums angelegt. Im Regelfall
hatte man dann drei Generationen, nämlich \tsTextQuoteG{Großvater},
\tsTextQuoteG{Vater} und \tsTextQuoteG{Sohn}. Dabei wurde die jeweils älteste
Generation überschrieben, und die anderen zwei Genreationen \tsTextQuoteG{rückten
    nach}.
\bigskip

Dadurch, dass der Backup Prozess längere Zeit in Anspruch nahm, wurde dies in
den späten Abendstunden bzw. Nachts durchgeführt. Hierbei ergab sich eine Art
von Lücke in der Datensicherung: Alles, was zum Zeitpunkt der Sicherung nicht
gespeichert war, ging verloren. Wenn also tagsüber ein Dokument angelegt und
vor der Datensicherung wieder gelöscht wurde, so hat man von diesem Dokument
keine Sicherung.

\subsection{Versionierung}

Um die letzte Lücke in der Datensicherung zu schliessen, verwendet man eine
sogenannte \gls{versionsverwaltung}. Diese merkt sich das Dokument und seinen
Werdegang. Sobald das Dokument zum ersten Mal gespeichert wurde, hat es die
initiale Version. Jede Änderung (und Speicherung!) an dem Dokument führt zu
einer nächsten Version. Das Löschen des Dokumentes führt zur letzten Version --
nicht mehr vorhanden. In der Versionisverwaltung ist das Dokument jedoch von
der initialen Version über die Zwischenversionen bis zur letzten Version noch
vorhanden.
