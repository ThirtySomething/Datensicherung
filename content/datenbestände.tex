%% -----------------------------------------------------------------------------
%% MIT License
%%
%% Copyright 2025 ThirtySomething
%%
%% Permission is hereby granted, free of charge, to any person obtaining a
%% copy of this software and associated documentation files (the "Software"),
%% to deal in the Software without restriction, including without limitation
%% the rights to use, copy, modify, merge, publish, distribute, sublicense,
%% and/or sell copies of the Software, and to permit persons to whom the
%% Software is furnished to do so, subject to the following conditions:
%%
%% The above copyright notice and this permission notice shall be included
%% in all copies or substantial portions of the Software.
%%
%% THE SOFTWARE IS PROVIDED "AS IS", WITHOUT WARRANTY OF ANY KIND, EXPRESS
%% OR IMPLIED, INCLUDING BUT NOT LIMITED TO THE WARRANTIES OF MERCHANTABILITY,
%% FITNESS FOR A PARTICULAR PURPOSE AND NONINFRINGEMENT. IN NO EVENT SHALL
%% THE AUTHORS OR COPYRIGHT HOLDERS BE LIABLE FOR ANY CLAIM, DAMAGES OR OTHER
%% LIABILITY, WHETHER IN AN ACTION OF CONTRACT, TORT OR OTHERWISE, ARISING
%% FROM, OUT OF OR IN CONNECTION WITH THE SOFTWARE OR THE USE OR OTHER
%% DEALINGS IN THE SOFTWARE.
%% -----------------------------------------------------------------------------

%% -----------------------------------------------------------------------------
%% Definition of terms
%% -----------------------------------------------------------------------------

\section{Datenbestände}

In diesem Abschnitt sollen die verschiedenen Arten der Datenbestände näher
beschrieben sowie voneinander abgegrenzt werden.

\subsection{Aktiver Datenbestand}

Unter einem aktiven Datenbestand versteht man die Daten, auf die jederzeit
ohne weiteres zugegriffen werden kann. Konkret geht es um die Dokumente, die
lokal auf einem Computer gesichert sind. Sobald man dem Computer arbeitet,
kann man aktiv auf diese Daten zugreifen.

\subsection{Passiver Datenbestand}

Im Gegensatz zum aktiven Datenbestand können auf Daten in einem passiven
Datenbestand nur indirekt zugegriffen werden. Beispielsweise müssen diese
Daten zuerst von einem Backupmedium wie einer externen Festplatte oder einem
Magnetband zuerst auf den Computer übertragen werden. Damit werden sie aus
dem passiven Datenbestand in den aktiven Datenbestand überführt.

\subsection{Backup}

Durch ein Backup wird eine Kopie des aktiven Datenbestandes auf einem externen
Speichermedium erstellt. Damit wurde der aktive Datenbestand des lokalen
Computers in einen passiven Datenbestand auf dem Backupmedium überführt.

\subsection{Synchronisation}

Bei der Synchronisation wird der aktive Datenbestand auf mehrere Geräte
erweitert. Konkret bedeutet das, dass die Daten von einem lokalen Computer auf
ein weiteres Gerät übertragen werden. Im Unterschied zum Backup kann aber
jederzeit auf dem weiteren Gerät ebenfalls auf die Daten zugegriffen werden
oder aber auch Änderungen vorgenommen werden. Die Daten werden nach der Änderung
synchronisiert und stehen dann auf allen Geräten mit dem gleichen Inhalt zur
Verfügung.
\bigskip

\tsTextBold{ACHTUNG:} Sollte ein und dasselbe Dokument auf mehreren Geräten
gleichzeitig geändert werden, kann es zu Konflikten kommen. Eine inhaltliche
Synchronisation der Daten ist nur mit besonderer kollaborativer Software
möglich!

\subsection{Versionierung}

Die Versionierung ist eine Form des Backups. Werden Änderungen an einem Dokument
durchgeführt, wird das alte Dokument zuerst als Version X gespeichert, bevor es
durch die aktuelle Version ersetzt wird.
