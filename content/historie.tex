%% -----------------------------------------------------------------------------
%% MIT License
%%
%% Copyright 2025 ThirtySomething
%%
%% Permission is hereby granted, free of charge, to any person obtaining a
%% copy of this software and associated documentation files (the "Software"),
%% to deal in the Software without restriction, including without limitation
%% the rights to use, copy, modify, merge, publish, distribute, sublicense,
%% and/or sell copies of the Software, and to permit persons to whom the
%% Software is furnished to do so, subject to the following conditions:
%%
%% The above copyright notice and this permission notice shall be included
%% in all copies or substantial portions of the Software.
%%
%% THE SOFTWARE IS PROVIDED "AS IS", WITHOUT WARRANTY OF ANY KIND, EXPRESS
%% OR IMPLIED, INCLUDING BUT NOT LIMITED TO THE WARRANTIES OF MERCHANTABILITY,
%% FITNESS FOR A PARTICULAR PURPOSE AND NONINFRINGEMENT. IN NO EVENT SHALL
%% THE AUTHORS OR COPYRIGHT HOLDERS BE LIABLE FOR ANY CLAIM, DAMAGES OR OTHER
%% LIABILITY, WHETHER IN AN ACTION OF CONTRACT, TORT OR OTHERWISE, ARISING
%% FROM, OUT OF OR IN CONNECTION WITH THE SOFTWARE OR THE USE OR OTHER
%% DEALINGS IN THE SOFTWARE.
%% -----------------------------------------------------------------------------

%% -----------------------------------------------------------------------------
%% Draft view about the history
%% -----------------------------------------------------------------------------

\section{Historie}

Früher wurden die Daten auf einem zentralen \gls{server} eines Unternehmens
gespeichert. Dieser Server verfügte über ein \gls{bandlaufwerk}, welches
regelmässig die Daten auf Band sicherte. Die Bänder wurden täglich gewechselt.
Optional wurde ein Satz der Bänder ausser Haus aufbewahrt, um im Falle eines
Brandes keinen Datenverlust zu erleiden.
\bigskip

Da sich die Technik weiterentwickelt hat, ist man dazu übergegangen, die Daten
nicht länger mit einem teuren Bandlaufwerk auf teure Bänder zu speichern,
sondern als Zielmedium eine \gls{festplatte} zu verwenden. Zum einen spielen
die gesunkenen Kosten für \gls{festplatte}n sowie deren gestiegene Kapazität
eine Rolle, zum anderen auch der Vorteil, dass die \gls{festplatte}n deutlich
schneller sind als \gls{bandlaufwerk}e.
\bigskip

Der neueste Trend bei der Datensicherung ist die Online-Sicherung. Hierbei
werden die Daten über das Internet in einer \gls{cloud} gesichert. Dieser
Trend wird zunehmend kritisch betrachtet, da man von der Preispolitik des
Anbieters abhängig ist. Zudem stellt sich die Frage, ob der Anbieter die
Daten ausschließlich für die Datensicherung verwendet oder ob er anderweitig
Nutzen daraus ziehen könnte. Wenn also sensible Unternehmensdaten gesichert
werden sollen, stellt sich die Frage, was ein möglicher Zugriff auf die Daten
durch den Anbieter für Konsequenzen haben könnte. Alternativ bedeutet das,
entweder keine Online Sicherung zu verwenden oder aber die Infrastruktur
hierfür selbst zu betreiben. Hier hat man nicht nur die Hardware outgesourced,
sondern auch noch die Dienstleistung der Serverbetreuung.
