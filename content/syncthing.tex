%% -----------------------------------------------------------------------------
%% MIT License
%%
%% Copyright 2025 ThirtySomething
%%
%% Permission is hereby granted, free of charge, to any person obtaining a
%% copy of this software and associated documentation files (the "Software"),
%% to deal in the Software without restriction, including without limitation
%% the rights to use, copy, modify, merge, publish, distribute, sublicense,
%% and/or sell copies of the Software, and to permit persons to whom the
%% Software is furnished to do so, subject to the following conditions:
%%
%% The above copyright notice and this permission notice shall be included
%% in all copies or substantial portions of the Software.
%%
%% THE SOFTWARE IS PROVIDED "AS IS", WITHOUT WARRANTY OF ANY KIND, EXPRESS
%% OR IMPLIED, INCLUDING BUT NOT LIMITED TO THE WARRANTIES OF MERCHANTABILITY,
%% FITNESS FOR A PARTICULAR PURPOSE AND NONINFRINGEMENT. IN NO EVENT SHALL
%% THE AUTHORS OR COPYRIGHT HOLDERS BE LIABLE FOR ANY CLAIM, DAMAGES OR OTHER
%% LIABILITY, WHETHER IN AN ACTION OF CONTRACT, TORT OR OTHERWISE, ARISING
%% FROM, OUT OF OR IN CONNECTION WITH THE SOFTWARE OR THE USE OR OTHER
%% DEALINGS IN THE SOFTWARE.
%% -----------------------------------------------------------------------------

%% -----------------------------------------------------------------------------
%% About Syncthing
%% -----------------------------------------------------------------------------

\section{Syncthing}

Das Programm \href{https://syncthing.net/}{Syncthing} ist eine Software, die
Daten zwischen verschiedenen Computern synchronisiert und je nach Konfiguration
auch noch versioniert. Für den Zweck der Dateisynchronisation
(Just-In-Time-Backup) mit Versionierung ist es die sprichwörtliche
\tsTextQuoteG{Eierlegende Wollmilchsau}.
\bigskip

Ein Weiteres Argument für die Software ist, dass sie als
\href{https://de.wikipedia.org/wiki/Open_Source}{Open Source}
zur Verfügung steht. Der Quellcode des Programmes ist öffentlich verfügbar,
kann jederzeit eingesehen werden, von jedem für seinen eigenen Zweck angepasst
werden und vieles mehr. Entscheidend sind jedoch zwei ganz wichtige Punkte:
Zum einen verhindert die öffentliche Verfügbarkeit des Quellcodes versteckte
Hintertüren, durch die jemand unbefugten Zugriff auf die Daten haben könnte.
Zum anderen entstehen zunächst einmal keine Kosten. Die Software kann kostenfrei
eingesetzt werden. Mögliche Kosten, die ein Dienstleister für die Einrichtung
der Software berechnet, sind hiervon ausgenommen.
\bigskip

Auf den folgenden Seiten wird ein Konzept vorgestellt, mit dem eine relativ
verlässliche Datensicherung erfolgen kann. Relativ verlässlich deswegen, denn
eine 100\%ige Sicherheit kann niemand gewährleisten. Das Konzept berücksichtigt
die bereits angesprochenen Punkte hinsichtlich der Redundanzen und der
Versionierung -- damit sollte ein hoher Grad an Datenverfügbarkeit gesichert
sein.
\bigskip

Die für dieses Konzept relevanten Funktionen von \gls{syncthing} sind:
\begin{itemize}
    \item Dateisynchronisation
    \item Dateiversionierung
\end{itemize}

\tsTextBold{Hinweis:} Das Programm \gls{syncthing} läuft auf einem Windows PC
primär im Hintergrund. Das bedeudet, dass sich nach der Einrichtung des Programmes
für den Anwender nichts ändert. Aber es ist für den Anwender auch nichts
sichtbar. Ergänzend dazu gibt es das Programm \gls{syncthingtray} -- ein
diskreter Aktivitätsindikator für \gls{syncthing}.
\bigskip

\tsTextBold{Anmerkung:} Da es sich bei Syncthing um ein Projekt aus dem
\href{https://de.wikipedia.org/wiki/Open_Source}{Open Source} Bereich handelt,
wird die Entwicklung der Software von freiwilligen kostenfrei durchgeführt.
Sollte das Konzept umgesetzt werden -- eine Spende für das Projekt wäre eine
tolle Geste.
