%% -----------------------------------------------------------------------------
%% MIT License
%%
%% Copyright 2025 ThirtySomething
%%
%% Permission is hereby granted, free of charge, to any person obtaining a
%% copy of this software and associated documentation files (the "Software"),
%% to deal in the Software without restriction, including without limitation
%% the rights to use, copy, modify, merge, publish, distribute, sublicense,
%% and/or sell copies of the Software, and to permit persons to whom the
%% Software is furnished to do so, subject to the following conditions:
%%
%% The above copyright notice and this permission notice shall be included
%% in all copies or substantial portions of the Software.
%%
%% THE SOFTWARE IS PROVIDED "AS IS", WITHOUT WARRANTY OF ANY KIND, EXPRESS
%% OR IMPLIED, INCLUDING BUT NOT LIMITED TO THE WARRANTIES OF MERCHANTABILITY,
%% FITNESS FOR A PARTICULAR PURPOSE AND NONINFRINGEMENT. IN NO EVENT SHALL
%% THE AUTHORS OR COPYRIGHT HOLDERS BE LIABLE FOR ANY CLAIM, DAMAGES OR OTHER
%% LIABILITY, WHETHER IN AN ACTION OF CONTRACT, TORT OR OTHERWISE, ARISING
%% FROM, OUT OF OR IN CONNECTION WITH THE SOFTWARE OR THE USE OR OTHER
%% DEALINGS IN THE SOFTWARE.
%% -----------------------------------------------------------------------------

%% -----------------------------------------------------------------------------
%% Definition of terms
%% -----------------------------------------------------------------------------

\section{Begriffe}

Bevor das Konzept vorgestellt wird, ist es wichtig, bestimmte Begriffe zu
klären, um im weiteren Verlauf Missverständnisse zu vermeiden.

\begin{itemize}
    \item \gls{backup} -- Sowohl die Sicherungskopie von Daten als auch der
          Prozess, um diese Sicherungskopie zu erstellen.
    \item \gls{bandlaufwerk} -- Ein Gerät, um Daten auf einem Magnetband
          zu sichern.
    \item \gls{cloud} -- Die Speicherung von Daten über das Internet bei
          einem Anbieter; wird auch als Filehosting bezeichnet.
    \item \gls{dateisynchronisation} -- Ein Prozess, um digitale Daten
          identisch an mehreren Orten zu speichern.
    \item \gls{festplatte} -- Konkret Festplattenlaufwerk, ein Gerät,
          auf dem auf magnetisierten Scheiben Daten gespeichert werden.
    \item \gls{generationenprinzip} -- Eine Strategie für die Datensicherung.
    \item Lokale Speicherung -- Die Speicherung von Daten auf einem lokalen
          Gerät.
    \item \gls{nas} -- Eine Art minimales Computersystem, welches
          \gls{festplatte}nspeicherplatz im Netzwerk für andere Netzwerkgeräte
          bereitstellt.
    \item \gls{pc} -- Ein Personal Computer, der im Regelfall mit dem
          Betriebssystem Windows verwendet wird.
    \item \gls{server} -- Ein
          Computersystem, welches im Regelfall 24/7 läuft und verschiedene
          Dienste im Netzwerk bereitstellt.
    \item \gls{syncthing} -- Ein Programm zur Synchronisation von Dateien
          zwischen mehreren Geräten.
    \item \gls{versionsverwaltung} -- Ein System zur Vorhaltung von mehreren
          Versionen eines Dokumentes.
    \item \gls{virtualisierung} -- Das ist ein Begriff aus der Computertechnik.
          Unterschieden wird zwischen Hardwarevirtualisierung und
          Softwarevirtualisierung. Bei der Hardwarevirtualisierung wird eine
          Hardware durch Software nachgebildet. Bei der Softwarevirutalisierung
          werden die Rahmenbedigungen für eine Software nachgebildet.
\end{itemize}
